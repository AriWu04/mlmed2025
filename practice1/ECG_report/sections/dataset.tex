\newpage
\section{Dataset}
\indent \indent In this practice, I use ECG Heartbeat Categorization Dataset from Kaggle as data source for labeled ECG records for practicing classification tasks purpose. This dataset consists of 5 types of heartbeats: Normal (N), Supraventricular (S), Ventricular (V), Fusion (F) and Unknown (Q), which are encoded to 0, 1, 2, 3 and 4 respectively. Each data contains 187 values of the heartbeat signal. 

\begin{enumerate}
    \item[\textbullet] Number of Samples: 109446
    \item[\textbullet] Number of Categories: 5
    \item[\textbullet] Sampling Frequency: 125Hz
    \item[\textbullet] Data Source: Physionet's MIT-BIH Arrhythmia Dataset {"cite"}
    \item[\textbullet] Classes: ['N': 0, 'S': 1, 'V': 2, 'F': 3, 'Q': 4]
\end{enumerate}

\begin{figure}[H]
    \centering
    \includesvg[width=1\columnwidth]{img/class_frequency.svg}    \caption{Frequency of each class in the ECG Heartbeat Dataset}
    \label{fig:ECGTrainFrequency}
\end{figure}

\indent As we can see from the histogram in Figure 1, the class distribution in the ECG heartbeat training dataset from Kaggle is severely skewed.  Class 0 leads the dataset with 72,471 samples, substantially outnumbering the other classes.  Class 1, Class 2, Class 3, and Class 4 have substantially fewer samples, with 2,223, 5,788, 641, and 6,431 occurrences, respectively.  This imbalance may influence model performance, necessitating approaches such as resampling or weighted loss functions to increase classification accuracy.

\indent To fix the class imbalance in the ECG heartbeat dataset and reduce the threat of overfitting, I applied the Synthetic Minority Over-sampling Technique (SMOTE) to balance the classes.

\indent SMOTE produces synthetic samples for minority classes by interpolating between existing instances, considerably boosting their representation without duplicating data. Unlike random oversampling, which may lead to overfitting by reproducing previous samples, or undersampling, which may result in loss of essential information, SMOTE helps establish a more general decision boundary. This feature assures the model learns substantial patterns from all classes, enhancing its capacity to classify minority heartbeat types appropriately.

\begin{figure}[H]
    \centering
    \includesvg[width=1\columnwidth]{img/SampleECG.svg}    
    \caption{Sample ECG Signals for Each Class}
    \label{fig:ECGTrainFrequency}
\end{figure}

\indent After resampling using SMOTE, the sample ECG signals for each class in the MIT-BIH Arrhythmia Dataset show distinct waveform patterns for different heartbeat categories. Each class exhibits unique characteristics in terms of amplitude and shape, reflecting the variations in cardiac activity. The resampling process ensures a balanced representation of all classes, allowing the model to learn from diverse ECG patterns. However, it is important to validate that the synthetic samples generated by SMOTE retain meaningful physiological features to avoid introducing unrealistic patterns that could affect model performance.

\clearpage