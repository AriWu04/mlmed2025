\section{Introduction}
\indent \indent
Electrocardiogram (ECG) signals are essential for identifying cardiovascular disease as they record the electrical activity of the heart. The precise classification of ECG heartbeats is crucial for the early identification of arrhythmias and other cardiac irregularities. Conventional procedures for identifying heartbeats depend on experienced medical professionals visually examining the ECG data, which are long, arbitrary, and prone to inaccuracies. Consequently, automated heartbeat categorization approaches have been created to increase the precision, effectiveness, and consistency of ECG analysis.

\indent In this report, I employ two machine learning techniques, XGBoost and convolutional neural networks (CNNs), to categorize ECG heartbeats. XGBoost, a powerful gradient boosting technique, is applied for its speed in processing structured data, while CNNs apply deep learning to uncover spatial patterns from ECG waveforms. By evaluating the performance of different models, I seek to evaluate their usefulness in improving automated heartbeat categorization, ultimately contributing to more reliable and efficient clinical diagnosis.

\section{Background}
\subsection{ECG}
\indent \indent An electrocardiogram (ECG) is a noninvasive diagnostic technique that measures the heart's electrical activity over time.  It consists of waveforms that indicate different stages of the cardiac cycle, including the P wave, QRS complex, and T wave.  Analyzing ECG data is critical for diagnosing arrhythmias, ischemia, and other cardiovascular problems.  However, manual interpretation of ECG data can be time-consuming and prone to mistakes.  Machine learning (ML) techniques have been progressively employed to automate ECG categorization, boosting diagnostic accuracy and efficiency.

\subsection{XGBoost}
\indent \indent XGBoost (Extreme Gradient Boosting) is an enhanced gradient boosting technique extensively used for classification jobs due to its speed, scalability, and strong predictive performance.  It generates an ensemble of decision trees using gradient-boosting techniques, reducing mistakes by repeatedly upgrading weak learners.  XGBoost is highly good at processing structured data with tabular information, making it suited for ECG classification when variables such as heart rate variability and waveform characteristics are retrieved and utilized as inputs.

\subsection{CNNs}
\indent \indent Convolutional Neural Networks (CNNs) are a family of deep learning models developed for processing grid-like input, such as pictures and time-series signals.  In ECG classification, CNNs automatically learn spatial and temporal characteristics from raw waveform data without needing manual feature extraction.  By using convolutional layers, pooling procedures, and fully connected layers, CNNs may successfully capture patterns in ECG data that correlate to distinct heartbeat types.  This ability makes CNNs particularly effective for ECG analysis, as they can learn complicated representations straight from the data, enhancing classification accuracy.

\indent This work employs both XGBoost and CNNs to evaluate their performance in ECG heartbeat classification and analyze their potential for enhancing automated diagnosis in the medical profession.

\clearpage
